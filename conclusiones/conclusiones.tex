\section{Conclusiones}
\subsection{Objetivos}
El objetivo de este trabajo, desde el primer momento, ha sido estudiar cómo se puede aproximar un sistema de RAG para otorgar contexto a un modelo, con la intención de mejorar la precisión con la que genere consultas JQL, ya que los modelos de lenguaje actuales, no parecen poseer mucho contexto sobre la generación de consultas JQL, además, si no son conocedores de la estructura de los proyectos, no pueden generar ciertas consultas complejas que sean útiles para los usuarios, por ello, se propuso realizar esta trabajo con fines de mejora, pero no realmente sujetos a un objetivo concreto, sino a la investigación y posible conclusión.

Si bien una ontología es una manera de representar un espacio de conocimiento, la documentación de JIRA no presenta una estructura demasiado compleja que representar con la semántica de una ontología. Podría considerarse que, por ende, no es realmente necesario el uso de una ontología para representar las reglas de JQL. Sin embargo, tras lo realizado en este trabajo, no sería complicado utilizar otra ontología que reuniese información más compleja y permitiese explotar la semántica de la ontología para mejorar la generación de consultas JQL.




A continuación se detallarán las distintas tecnologías que serán estudiadas durante este TFG. Cabe recalcar que varias de estas distintas tecnologías propuestas, como los Knowledge Graphs (KGs) o las ontologías, han requerido de un estudio previo para poder ser implementadas en el proyecto.

Independientemente de los resultados que se obtengan con cada una de ellas, es necesario tener en cuenta el proceso de familiarización con las mismas, así como el tiempo invertido en su estudio y posterior implementación para un desempeño óptimo.

La parte de los grafos de conocimiento supone un reto, ya que es uno de los ejes del trabajo que más dudas ha generado.




\subsection{Conclusiones personales}
Este trabajo, aunque ha tenido sus altibajos, ha sido una experiencia que calificaría como altamente positiva. En un principio, la idea que se me planteó de adentrarme en un mundo que no conocía, como es el de los grafos de conocimiento o las ontologías, me pareció bastante llamativa por el simple hecho de que iba a aprender algo nuevo, además, juntándolo con el estado del arte de los modelos de lenguaje, parecía que iba a ser realmente enriquecedor, y así ha sido. Ha habido momentos de mayor frustración, donde parecía que la implementación de las diferentes propuestas estaba un poco "cogida con pinzas", pero la investigación es así, y es lo que la hace interesante. Si algo saco en claro de este trabajo, es que la investigación es un camino que no es sencillo, pero es muy satisfactorio y no podría estar más agradecido de haber hecho el trabajo con Arkaitz, Mikel y Unai, que me han ayudado en todo momento y que me han hecho apreciar lo que estaba haciendo.

\subsection{Trabajo futuro}
En cuanto al trabajo posible realizable en este proyecto, se podrían proponer distintas mejoras, como la creación de un grafo más complejo y con una mejor representación de la información o una ontología que cubriese de mejor manera los conceptos enteros de JIRA, no solo JQL, ya que este no parece contener tanta semántica como para que una ontología parezca realmente útil. Mencionar que, durante el transcurso del trabajo, se pidió un proyecto de Hazitek para obtener financiación y desarrollar el proyecto más allá de su estado actual, con una posible mejora subsecuente.