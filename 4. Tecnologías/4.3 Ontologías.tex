\subsection{Ontologías}
Una ontología en el campo de la informática es una especificación formal y explícita de una conceptualización. En otros términos, una ontología es una representación de un conjunto de conceptos dentro de un dominio y las relaciones entre esos conceptos. 

El lenguaje usado para definir ontologías es el \textit{Web Ontology Language} (OWL), que es un lenguaje de marcado semántico para publicar y compartir ontologías en la web. OWL es desarrollado por el \textit{World Wide Web Consortium} (W3C) y es una extensión de \textit{Resource Description Framework} (RDF), que es un modelo de datos basado en grafos para describir recursos web.

Como posible base de conocimiento relevante para la generación de consultas JQL se propone crear una ontología que represente las reglas que existen en las consultas JQL. La información que se pretende representar en la ontología se ha extraído directamente de la documentación oficial de Jira, brindada por Atlassian, donde se detallan las reglas que se deben seguir para la creación de consultas JQL \cite{jiradocs}. Esta ontología serviría para interpretar las reglas que hay que seguir al generar consultas JQL, además, consta de ejemplos en cada una de las clases definidas, que ayuda a comprender mejor el funcionamiento de las reglas.

Esta ontología ha sido desarrollada utilizando el software \textit{Protégé} \cite{protege}, una herramienta de código abierto para la creación de ontologías desarrollada por la Universidad de Stanford.