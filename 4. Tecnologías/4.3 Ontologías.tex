\subsection{Ontologías}
Una ontología es una representación de un conjunto de conceptos dentro de un dominio y las relaciones entre esos conceptos. 

La utilidad de las ontologías radica en que permiten a los sistemas de información compartir conocimiento de manera más eficiente, ya que proporcionan una estructura común y unificada para la descripción de los datos. Además, las ontologías permiten la interoperabilidad entre sistemas, ya que facilitan la comunicación entre ellos al tener una representación común del conocimiento. A día de hoy, son objeto de estudio en combinación con los modelos de lenguaje, explorando la posibilidad de mejorar la interpretación de los datos y la generación de consultas.

El lenguaje usado para definir ontologías es Web Ontology Language (OWL), que es un lenguaje de representación del conocimiento (KR, por sus siglas en inglés) para publicar y compartir ontologías en la web. OWL es desarrollado por el World Wide Web Consortium (W3C)\footnote{https://www.w3.org/} y tanto este como RDFS son lenguajes que proporcionan una forma de referirse a los esquemas de los datos RDF.
