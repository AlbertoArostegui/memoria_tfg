\subsection{Ontologías}
Una ontología en el campo de la informática es una especificación formal y explícita de una conceptualización. En otros términos, una ontología es una representación de un conjunto de conceptos dentro de un dominio y las relaciones entre esos conceptos. 

La utilidad de las ontologías radica en que permiten a los sistemas de información compartir conocimiento de manera más eficiente, ya que proporcionan una estructura común y unificada para la representación de los datos. Además, las ontologías permiten la interoperabilidad entre sistemas, ya que facilitan la comunicación entre ellos al tener una representación común del conocimiento. A día de hoy, son objeto de estudio en combinación con los modelos de lenguaje, explorando la posibilidad de mejorar la interpretación de los datos y la generación de consultas.

El lenguaje usado para definir ontologías es el \textit{Web Ontology Language} (OWL), que es un lenguaje de marcado semántico para publicar y compartir ontologías en la web. OWL es desarrollado por el \textit{World Wide Web Consortium} (W3C) y es una extensión de \textit{Resource Description Framework} (RDF), que es un modelo de datos basado en grafos para describir recursos web.

% TODO: cambiar y mover esto a otra sección
