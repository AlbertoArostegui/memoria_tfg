\subsection{Grafos de conocimiento}
Los grafos de conocimiento, conocidos más comúnmente como \textit{knowledge graphs} en inglés, son estructuras de datos que representan información de manera semántica. Estos grafos se componen de nodos y aristas, donde los nodos representan entidades y las aristas representan relaciones entre estas entidades. Los grafos de conocimiento son especialmente útiles para representar información estructurada y para realizar consultas complejas sobre esta información.

Los grafos de conocimiento se popularizaron con la creación de \textit{Google Knowledge Graph} en 2012, un servicio de Google que proporciona información adicional en los resultados de búsqueda. Este servicio se basa en un grafo de conocimiento que contiene información sobre una gran cantidad de entidades y relaciones entre estas entidades. Desde entonces, los grafos de conocimiento han sido utilizados en una gran variedad de aplicaciones, como la búsqueda semántica, las redes sociales y la recomendación de contenido.

A día de hoy, existen dos modelos de grafo de conocimiento más populares: \textit{Resource Description Framework} (RDF) y \textit{Property Graph Model}. El modelo RDF es un modelo de grafo de conocimiento basado en la web que se utiliza para representar información semántica en la web. En este tipo de grafo, los nodos representan recursos atómicos y las aristas representan relaciones entre estos, además, estos grafos son dirigidos, en este tipo de grafos la unidad de discurso se considera la unión de dos nodos por una relación. Por otro lado, el modelo de \textit{Property Graph} es un modelo de grafo de conocimiento que tiende a usar nodos como objetos de un dominio, como tipos de datos ricos, donde estos mismos pueden considerarse la unidad de discurso, como se explica en esta cuestión de Stack Overflow~\cite{rdfVsPropertyGraph}.

En el ámbito de la inteligencia artificial, los grafos de conocimiento son especialmente útiles para mejorar el rendimiento de los modelos de lenguaje. Al combinar la información de un grafo de conocimiento con la información de un modelo de lenguaje, se pueden generar respuestas más precisas y acordes al contexto. Esto es una ventaja sobre los \textit{Embeddings} ya que proporcionan un mayor contexto y una mayor precisión en las respuestas generadas. Son especialmente útiles en contextos donde es relevante mucha información interrelacionada, como en ámbitos de salud o medicina, donde es realmente útil tener en cuenta toda la información disponible para dar una respuesta precisa.