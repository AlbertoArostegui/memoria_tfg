\subsection{Knowledge Graphs}
Los Knowledge Graphs son estructuras de datos que representan información de manera semántica. Estos grafos se componen de nodos y aristas, donde los nodos representan entidades y las aristas representan relaciones entre estas entidades, como un grafo tradicional. Los Knowledge Graphs difieren de estos en que las entidades están anotadas con términos de una ontología, por lo que respetan la semántica de la misma.  

Los Knowledge Graphs se popularizaron con la creación de Google Knowledge Graph en 2012, un servicio de Google que proporciona información adicional en los resultados de búsqueda. Este servicio se basa en un Knowledge Graph que contiene información sobre una gran cantidad de entidades y relaciones entre estas entidades. Desde entonces, los grafos de conocimiento han sido utilizados en una gran variedad de aplicaciones, como la búsqueda semántica, las redes sociales y la recomendación de contenido.

A día de hoy, existen dos modelos de Knowledge Graph más populares: Resource Description Framework (RDF) y Property Graph Model. El modelo RDF es un estándar abierto desarrollado por W3C\footnote{https://www.w3.org/RDF/} para la representación semántica de recursos en la Web. Está formado por declaraciones de la forma sujeto-predicado-objeto, conocidas como tripletas. Por otro lado, los Property Graphs buscan representar la información de manera más expresiva, permitiendo la definición de propiedades en las aristas y los nodos, sin embargo, carecen de la interoperabilidad del modelo RDF por ser desarrollado por W3C. 

En el ámbito de la inteligencia artificial, los Knowledge Graphs son especialmente útiles para mejorar el rendimiento de los modelos de lenguaje. Al combinar la información de un knowledge graph con la información de un modelo de lenguaje, se pueden generar respuestas más precisas y acordes al contexto. Ésto es una ventaja sobre los Embeddings ya que proporcionan un mayor contexto y una mayor precisión en las respuestas generadas. Son especialmente útiles en contextos donde es relevante mucha información interrelacionada, como en ámbitos de salud o medicina, donde es realmente útil tener en cuenta toda la información disponible para dar una respuesta precisa.