\section{Introducción}
En la actualidad, la gestión eficiente de proyectos es esencial para las empresas de desarrollo de software que buscan mantenerse competitivas y ofrecer productos de alta calidad. En este contexto, las herramientas que facilitan la organización, planificación y control de recursos se convierten en elementos clave. Este trabajo, realizado en colaboración con la empresa LKS Next-GobTech, que se dedica al desarrollo de software, con gran enfoque en la innovación, explora estas tecnologías.

El proyecto se basa en mejorar las capacidades de JIRA, una herramienta ampliamente utilizada para la gestión de proyectos, mediante la implementación de tecnologías avanzadas de procesamiento del lenguaje natural (PLN). En el ámbito de la gestión de proyectos, la persona encargada de la supervisión y control de las tareas necesita acceder a información específica sobre el estado de los proyectos, como el número de incidencias abiertas, el tiempo estimado para completar una tarea o el progreso de un proyecto en particular. Para obtener esta información, JIRA ofrece un lenguaje de consulta llamado JIRA Query Language (JQL), que permite realizar consultas avanzadas sobre los datos almacenados en la plataforma. Sin embargo, un gestor de proyectos no siempre está familiarizado con la sintaxis y estructura de JQL, lo que puede dificultar la obtención de la información deseada. Es por esto que, partiendo del trabajo previo de un compañero, que desarrolló un asistente conversacional capaz de generar consultas JQL a partir de preguntas en lenguaje natural, este proyecto explora la integración de una arquitectura de Generación Aumentada por Recuperación (RAG, por sus siglas en inglés) para aumentar la precisión y utilidad de las respuestas ofrecidas por el sistema, buscando mejorar la interacción y eficiencia en el uso de JIRA.

La estructura del trabajo abarca desde la contextualización de las herramientas y metodologías utilizadas en la gestión de proyectos hasta la descripción detallada del desarrollo e implementación de la solución propuesta. Se analiza cómo las tecnologías de modelos de lenguaje de gran escala (LLM) y la generación aumentada por recuperación pueden integrarse en el flujo de trabajo existente para mejorar la interacción y eficiencia en el uso de JIRA.

\subsection{Objetivos del trabajo}
Para la realización de este trabajo, se han establecido unos objetivos junto con LKS Next-GobTech.

El objetivo principal de este trabajo es realizar un estudio sobre la potencial mejora de precisión en la generación de consultas JQL que ofrecen los modelos de lenguaje natural de hoy en día. Para ello, se propone la integración de una arquitectura de Generación Aumentada por Recuperación (RAG) en un asistente conversacional previamente desarrollado, que permita mejorar la interacción con JIRA y la eficiencia en la obtención de información relevante.

De cara a la realización de las pruebas y la evaluación de resultados obtenidos se ha desarrollado en Python un entorno de pruebas que permita evaluar la eficacia de la solución propuesta. Este entorno de pruebas se ha diseñado para realizar pruebas de rendimiento y comparar los resultados obtenidos con y sin la integración de RAG. Además, permite de manera sencilla la evaluación de más modelos de lenguaje y diferentes técnicas de recuperación de información.

