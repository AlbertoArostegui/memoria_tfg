\subsection{Presupuesto}
A lo largo de esta subsección se detalla el presupuesto necesario para el estudio transcurrido en este trabajo de fin de grado. Se incluyen los costes de los recursos humanos, los costes de los recursos materiales y los costes de los recursos software.

\subsubsection{Costes de software}
El estudio de este trabajo se ha realizado con herramientas de software de código abierto en la medida de lo posible, si bien también se ha hecho uso de llamadas de API de OpenAI.

Como editores de código y ontologías, se han utilizado Visual Studio Code y Protégé respectivamente, ambos de código abierto y gratuitos. Para la gestión de versiones se ha utilizado Git, también de código abierto y gratuito.


\subsubsection{Costes de mano de obra}
Los costes de mano de obra se han calculado en base a las horas de trabajo invertidas en el proyecto y el salario medio de un ingeniero informático en España, además del coste de dos profesores adjuntos de la Universidad del País Vasco y un tutor de la empresa LKS Next-GobTech.
En el caso del ingeniero informático, se asume un rol de programador junior, con un salario medio de 21000€ brutos anuales, al cual habría que descontarle impuestos y retenciones. 
En el caso del prfesor adjunto, según la página de la UPV/EHU, el salario medio es de 34144€ brutos anuales. 

\subsubsection{Costes de hardware}
Para la realización de este trabajo se han utilizado tanto un portátil Lenovo Thinkpad T14, durante la estancia en LKS Next-GobTech, como un ordenador de sobremesa montado por partes, con los siguientes componentes:

\begin{itemize}
    \item Procesador: AMD Ryzen 7 7800X3D
    \item Tarjeta gráfica: NVIDIA GeForce RTX 2080 Super 8GB VRAM
    \item Memoria RAM: 32 GB DDR5
\end{itemize}

Valorado, en el momento de compra, en 1500€.

