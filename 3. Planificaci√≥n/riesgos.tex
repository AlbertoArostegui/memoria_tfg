\subsection{Gestión de riesgos}
Durante la realización de un proyecto se han de tener en cuenta los posibles riesgos que puedan surgir, por lo que es necesario un plan de gestión de riesgos, donde se detalla la prevención y mitigación de los mismos. En esta subsección se detallan los riesgos que se han identificado y cómo se han gestionado.

\tabla{Baja por enfermedad}{
    \textbf{Riesgo}: Baja por enfermedad del alumno.
}{
    \textbf{Probabilidad}: Baja
}{
    \textbf{Impacto}: Alto
}{
    \textbf{Prevención}: El alumno se asegurará de llevar una vida saludable y de no excederse en el trabajo para evitar enfermedades.
}{
    \textbf{Mitigación}: En caso de enfermedad, se intentará seguir trabajando en la medida de lo posible y se pedirá ayuda a los tutores.
}

\tabla{Problemas técnicos}{
    \textbf{Descripción}: Problemas técnicos con el hardware o software, ya sea un mal funcionamiento o la total incapacidad de uso.
}{
    \textbf{Probabilidad}: Media
}{
    \textbf{Impacto}: Medio
}{
    \textbf{Prevención}: Se realizarán copias de seguridad periódicas y se mantendrán actualizados los sistemas.
}{
    \textbf{Mitigación}: En caso de problemas técnicos, se intentará solucionar el problema lo antes posible y se pedirá ayuda a los tutores, se puede continuar trabajando en el otro equipo en caso de que uno falle.
}

\tabla{Falta de motivación}{
    \textbf{Descripción}: Falta de motivación para continuar con el proyecto, dejando el trabajo a medias y aplazando lo planeado.
}{
    \textbf{Probabilidad}: Baja
}{
    \textbf{Impacto}: Muy alto
}{
    \textbf{Prevención}: Se realizarán reuniones semanales con los tutores para mantener la motivación y el interés en el proyecto.
}{
    \textbf{Mitigación}: En caso de falta de motivación, se intentará buscar nuevas formas de motivación y se pedirá ayuda a los tutores.
}

\tabla{Mala acotación del alcance}{
    \textbf{Descripción}: Mala acotación del alcance del proyecto durante la planificación, lo que puede llevar a un desbordamiento de las tareas y a un retraso en la entrega.
}{
    \textbf{Probabilidad}: Media
}{
    \textbf{Impacto}: Alto
}{
    \textbf{Prevención}: Se realizarán reuniones semanales con los tutores para revisar el alcance del proyecto y se mantendrá actualizado el plan de trabajo.
}{
    \textbf{Mitigación}: En caso de mala acotación del alcance, se intentará redefinir el alcance del proyecto y se pedirá ayuda a los tutores.
}