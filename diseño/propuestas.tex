\subsection{Propuestas}
Se propone utilizar arquitecturas \aclink{RAG} para mejorar la precisión, ofreciendo al modelo información sobre la generación de consultas \aclink{JQL}. La idea detrás de esto es que, al tener un modelo de recuperación que pueda acceder a una base de conocimiento, el modelo de generación pueda generar respuestas más precisas y acordes al contexto proporcionado. Además, se propone un nuevo conjunto de datos:

Si bien el conjunto de datos inicial era robusto, se ha propuesto un nuevo conjunto, de 100 preguntas, que busca, no solo tener más datos, sino hacerlos más diversos y cambiar en cierto modo las preguntas para cubrir el máximo número de casos posible. Este conjunto de datos se ha pensado durante el desarrollo y las diferentes pruebas lanzadas y también se ha usado como apoyo un dataset existente en \textit{Hugging Face}~\cite{datasetHF}.

A continuación, se describirán las distintas alternativas propuestas para mejorar la precisión del modelo JiraGPT Next.

\subsubsection{Ontología}
Durante el inicio de este trabajo se consultaron artículos como \textit{Sequeda et al.}~\cite{sequeda2023benchmark}, que exploraban la posibilidad de utilizar ontologías en el prompt para mejorar la interpretación de los datos y la generación de consultas SQL, logrando resultados prometedores. Partiendo de esta idea, se propone entonces crear una ontología que represente las reglas que existen en las consultas JQL. La información que se pretende representar en la ontología se ha extraído directamente de la documentación oficial de Jira, brindada por Atlassian, donde se detallan las reglas que se deben seguir para la creación de consultas JQL \cite{jiradocs}. Esta ontología serviría para interpretar las reglas que hay que seguir al generar consultas JQL, además, consta de ejemplos en cada una de las clases definidas, que ayuda a comprender mejor el funcionamiento de las reglas.

Para crear esta ontología se ha utilizado el software \textit{Protégé}~\cite{protege}, una herramienta de código abierto para la creación de ontologías desarrollada por la Universidad de Stanford. La ontología se ha desarrollado siguiendo el estándar \textit{Web Ontology Language} (OWL), que es un lenguaje de marcado semántico para publicar y compartir ontologías en la web. OWL es desarrollado por el \textit{World Wide Web Consortium} (W3C) y es una extensión de \textit{Resource Description Framework} (RDF).
