\begin{itshape}
    \textbf{Resumen:} \\
        Este proyecto consiste en la realización de un estudio que evalúe la potencial mejora de precisión en las respuestas de un LLM implementando técnicas de RAG. Con tal de mejorar la experiencia de usuario dentro de la aplicación para gestión de proyectos Jira, se dispone que una herramienta que hace uso de LLMs para generar consultas JQL. Durante este trabajo se hará un estudio que buscará mejorar la precisión de estas respuestas haciendo uso de bases de conocimiento para aportar contexto al modelo. Se van a a implementar una ontología, una base de datos vetorial con embeddings y un Knowledge Graph.

        Con tal de evaluar la precisión obtenida con cada una de las propuestas se ha creado un benchmark que evalúe la respuesta del modelo para cada pregunta,

    
    \textbf{Palabras Clave:} RAG, LLM, Jira, JQL, Ontología, Knowledge Graph, Embeddings
\end{itshape}
\newpage

\begin{itshape}
    \textbf{Laburpena:} \\
    Euskera

    
    \textbf{Gako-hitzak:}
\end{itshape}
\newpage

\newpage

\begin{itshape}
    \textbf{Abstract:} \\
    English


    \textbf{Key Words:}
\end{itshape}
