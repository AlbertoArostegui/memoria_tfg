\section{Anexo I}

\subsection{Conjunto de preguntas}
Las 100 preguntas utilizadas para la evaluación de las distintas propuestas de mejora y la consulta JQL que se utiliza para comprobar las incidencias esperadas.

\begin{center}
    \begin{longtable}{ | p{1cm} | p{8cm} | p{6cm} | }
        \hline
        \textbf{ID} & \textbf{Pregunta} & \textbf{Consulta JQL} \\
        \hline
        \endfirsthead

        \hline
        \textbf{ID} & \textbf{Pregunta} & \textbf{Consulta JQL} \\
        \hline
        \endhead

        \hline
        \multicolumn{3}{|r|}{Continúa en la siguiente página} \\
        \hline
        \endfoot

        \hline
        \endlastfoot

        1 & Muestra las incidencias en progreso & status = 'En progreso' \\
        \hline
        2 & Muestra las incidencias en progreso y sus horas acumuladas & status = 'En progreso'   \\
        \hline
        3 & Muestra las incidencias que ya estan entregadas a cliente  & status = 'Entregado'   \\
        \hline
        4 & Muestra las incidencias que se han pasado de horas y por cuánto  & workratio > 100   \\
        \hline
        5 & Muestra las incidencias que deberían acabar en agosto de 2023  & due >= '2023-08-01' AND due <= '2023-08-31' \\
        \hline
        6 & Muestra las incidencias sin cerrar ni resolver  & status not in (Resuelto, Cerrado)   \\
        \hline
        7 & Muestra quién tiene menos carga de trabajo  &  ORDER BY assignee ASC, updated DESC \\
        \hline
        8 & Muestra quién tiene menos carga de trabajo en MFM  & project = 'MFM' \\
        \hline
        9 & Cuántos tickets se han resuelto en agosto de 2023? & status changed to Resuelto DURING (´2023-08-01´, ´2023-08-31´) \\
        \hline
        10 & Cuántos tickets se han resuelto en agosto de 2023? & status changed to Resuelto DURING (´2023-08-01´, ´2023-08-31´) \\
        \hline
        11 & Muéstrame todas las incidencias abiertas asignadas a Alberto Aróstegui García  &   assignee = ´Alberto Aróstegui García´ AND status = Abierto \\
        \hline
        12 & Muéstrame todas las incidencias con estado Abierto asignadas a Alberto Aróstegui García  &   assignee = ´Alberto Aróstegui García´ AND status = Abierto \\
        \hline
        13 & ¿Cuántos tickets sin resolver hay? &   status != Resuelto \\
        \hline
        14 & Encuentra las incidencias asignadas a Alberto Aróstegui García. &   assignee = ´Alberto Aróstegui García´ \\
        \hline
        15 & ¿Cuántas incidencias se cerraron en agosto de 2023? & status changed to Cerrado DURING (´2023-08-01´, ´2023-08-31´) \\
        \hline
        16 & ¿ que incidencias fueron cerradas durante agosto de 2023? &   status changed to Cerrado DURING (´2023-08-01´, ´2023-08-31´) \\
        \hline
        17 & Muestra todas las incidencias con alta prioridad  & priority = High   \\
        \hline
        18 & Dame una lista de las incidencias cuyo estado cambio en agosto de 2023 . & status CHANGED DURING (´2023-08-01´, ´2023-08-31´) \\
        \hline
        19 & Encuentra todos los tickets con ´base de datos´ en el resumen . & summary ~ ´base de datos´   \\
        \hline
        20 & ¿Cuántos tickets están resueltos ? &   status = Resuelto \\
        \hline
        21 & Muestra todas las incidencias reportadas por Alberto Aróstegui García  & reporter = ´Alberto Aróstegui García´   \\
        \hline
        22 & Muestra el progreso de la incidencia con el campo personalizado Custom\_Id con valor GPT4-30 & Custom\_Id ~ GPT4-30   \\
        \hline
        23 & ¿Cuáles son las incidencias creadas en agosto de 2023 ? & created >= '2023-08-01' AND created <= 2023-08-31 \\
        \hline
        24 & ¿Cuáles son las incidencias creadas en agosto de 2023 ? Ordenalas de mayor a menor prioridad & created >= '2023-08-01' AND created <= '2023-08-31' ORDER BY priority DESC \\
        \hline
        25 & ¿Cuántos bugs se han reportado? & issuetype = Bug   \\
        \hline
        26 & Dame todas las incidencias en la fase de ´Resuelto´. &  status = 'Resuelto' \\
        \hline
        27 & Muéstrame todas las incidencias reabiertas  & status = Reabierto   \\
        \hline
        28 & Muéstrame todas las incidencias con estado Reabierto'  & status = Reabierto   \\
        \hline
        29 & Muestra todos los tickets que vencen en agosto de 2023 . & due >= '2023-08-01' AND due <= '2023-08-31' \\
        \hline
        30 & ¿Qué tickets terminan después del 1 de septiembre de 2023 ? & due > '2023-09-01'   \\
        \hline
        31 & Encuentra todas las incidencias movidas a ´Validado´ en agosto de 2023  & status changed to Validado DURING (´2023-08-01´, ´2023-08-31´) \\
        \hline
        32 & Dame las incidencias resueltas en agosto de 2023  & status changed to Resuelto DURING (´2023-08-01´, ´2023-08-31´) \\
        \hline
        33 & ¿Cuántos tickets están en fase de ´En Progreso´ ? & status = ´En progreso´   \\
        \hline
        34 & Lista todos los tickets creados en agosto de 2023  & created >= '2023-08-01' AND created <= '2023-08-31'   \\
        \hline
        35 & Muéstrame las incidencias abiertas más antiguas . & status = Abierto and ORDER BY created ASC \\
        \hline
        36 & Muéstrame las incidencias con estado Abierto' más antiguas .' & status = 'Abierto' ORDER BY created ASC \\
        \hline
        37 & Lista todos los tickets con la etiqueta ´seguridad´ y dame un resumen . & labels = 'seguridad' \\
        \hline
        38 & ¿Cuántas incidencias se han movido a ´En progreso´ en agosto de 2023 ? &  AND status changed to ´En progreso´ DURING (´2023-08-01´, ´2023-08-31´) \\
        \hline
        39 & ¿Cuáles son las incidencias de máxima prioridad asignadas a Alberto Aróstegui García ? &  AND assignee = 'Alberto Aróstegui García' AND priority = Highest \\
        \hline
        40 & Lista todas las incidencias cerradas en agosto de 2023  &  AND status changed to Cerrado DURING (´2023-08-01´, ´2023-08-31´) \\
        \hline
        41 & ¿Cuántas incidencias nuevas se han agregado en agosto de 2023? &  AND created >= '2023-08-01' AND created <= '2023-08-31' \\
        \hline
        42 & Encuentra todas las incidencias con ´rendimiento´ en su descripción . & description ~ rendimiento \\
        \hline
        43 & Encuentra todos los tickets movidos a ´Cerrado´ en agosto de 2023 . & status changed to Cerrado DURING (´2023-08-01´, ´2023-08-31´) \\
        \hline
        44 & Muestra todas las incidencias que están vencidas  & due < now() and status != Cerrado \\
        \hline
        45 & Lista todos los tickets en los que alguien ha votado . & voter is not empty   \\
        \hline
        46 & ¿Cuántas incidencias están en la fase ´Abierto´ ? &  AND status = Abierto \\
        \hline
        47 & ¿Cuántas incidencias están abiertas ? &  AND status = Abierto \\
        \hline
        48 & Recupera las incidencias que tengan que ver con Gitlab  & summary ~ 'Gitlab' OR description ~ 'Gitlab'    \\
        \hline
        49 & Muestra las incidencias sin asignar  &  AND assignee is EMPTY \\
        \hline
        50 & Cuantas incidencias asignadas hay ? &  AND assignee is not EMPTY \\
        \hline
        51 & Muestrame las incidencias abiertas  & status not in (Resuelto, Cerrado)   \\
        \hline
        52 & Que incidencias de maxima prioridad  &  and priority = Highest \\
        \hline
        53 & Selecciona las incidencias sin resolver de esta semana  & created >= startOfWeek() and status != Resuelto   \\
        \hline
        54 & Muestra todas las incidencias con enlaces externos  & attachments is not EMPTY \\
        \hline
        55 & Selecciona las incidencias que vencen en los proximos 3 dias  & due <= endOfDay('+3d') and status != Resuelto   \\
        \hline
        56 & Muestra las incidencias en las que se esta trabajando ahora mismo  & status = 'En Progreso'   \\
        \hline
        57 & Muestra las incidencias creadas esta semana  &  and created >= startOfWeek() \\
        \hline
        58 & Muestra las incidencias creadas en los ultimos 7 dias  & created >= -7d()  \\
        \hline
        59 & Muestrame las incidencias que vencen el 11 de agosto de 2023  & due = '2023-08-11' \\
        \hline
        60 & Dame las incidencias que contengan etiquetas &  labels is not EMPTY \\
        \hline
        61 & Muestrame las incidencias asignadas a Alberto Aróstegui García que tengan etiquetas  & assignee = 'Alberto Aróstegui García' and labels is not EMPTY   \\
        \hline
        62 & Dame las incidencias asignadas a mi  que contengan enlaces externos & assignee = currentUser() and attachments is not EMPTY \\
        \hline
        63 & Dame las incidencias asignadas a mi  & assignee = currentUser() \\
        \hline
        64 & Muestrame las incidencias que estan en un sprint abierto & sprint in openSprints() \\
        \hline
        65 & Que incidencias estan en un sprint ya cerrado & sprint in closedSprints() \\
        \hline
        66 & ¿Cuáles son los ítems que tienen una fecha de vencimiento entre el 1 de agosto de 2023 y el 31 de agosto de 2023 ? & due >= '2023-08-01' AND due <= '2023-08-31' \\
        \hline
        67 & ¿En que proyectos ha votado al menos una persona? & voter is not EMPTY \\
        \hline
        68 & ¿Qué inicidencias tengo yo asignadas ? & assignee = currentUser() \\
        \hline
        69 & ¿Qué incidencias están resueltas ahora mismo ? & status = Resuelto \\
        \hline
        70 & ¿Qué tickets se cerraron durante agosto de 2023? & status changed to Cerrado DURING (´2023-08-01´, ´2023-08-31´) \\
        \hline
        71 & ¿Qué tickets cambiaron de estado durante agosto de 2023? & status changed DURING (´2023-08-01´, ´2023-08-31´) \\
        \hline
        72 & ¿Qué incidencias fueron creadas el último mes ? & created >= startOfMonth(-1) \\
        \hline
        73 & ¿Cuántas incidencias  están cerradas pero no resueltas? & status = Cerrado and status != Resuelto \\
        \hline
        74 & Muestrame las incidencias que se han movido a Resuelto en agosto de 2023 & status changed to Resuelto DURING (´2023-08-01´, ´2023-08-31´) \\
        \hline
        75 & Dame las incidencias que han sido actualizadas en las ultimas 24 horas & updated >= -24h() \\
        \hline
        76 & ¿Que incidencias vencen hoy ? & due <= endOfDay() \\
        \hline
        77 & ¿Cuantas incidencias estan abiertas pero no asignadas? & status = Abierto and assignee is EMPTY \\
        \hline
        78 & ¿Qué incidencias fueron resueltas en los últimos 30 días ? & status changed to Resuelto DURING (´-30d´, now()) \\
        \hline
        79 & ¿Cuántas incidencias se han movido a ´En progreso´ en los últimos 7 días ? & status changed to ´En progreso´ DURING (´-7d´, now()) \\
        \hline
        80 & ¿Qué incidencias  están en un sprint y no están asignadas? & sprint in openSprints() and assignee is EMPTY \\
        \hline
        81 & ¿Que incidencias tienen enlaces externos? & attachments is not EMPTY \\
        \hline
        82 & Lista todas las incidencias que contengan la palabra ´urgente´  & summary ~ ´urgente´ \\
        \hline
        83 & Dame las incidencias de prioridad alta que todavia no estan asigndas & priority = High and assignee is EMPTY \\
        \hline
        84 & Muestramen las incidencias cerradas el último trimestre  & status changed to Cerrado DURING (´-3M´, now()) \\
        \hline
        85 & Dame las incidencias de baja prioridad que estan en progreso & priority = Low and status = ´En progreso´ \\
        \hline
        86 & ¿Qué incidencias tienen un campo personalizado con valor ´GPT4-30´? & Custom\_Id ~ GPT4-30 \\
        \hline
        87 & ¿Cuántas incidencias se han creado en los últimos 7 días ? & created >= -7d() \\
        \hline
        88 & Muestra las incidenciasque están marcadas como críticas y aún están abiertas & priority = Highest and status = Abierto \\
        \hline
        89 & Dame las incidencias que han cambiado de prioridad & priority CHANGED \\
        \hline
        90 & ¿Cuántas incidencias relacionadas con fallos hay ? & issuetype = Bug \\
        \hline
        91 & Dame el total de horas registradas en las incidencias tipo Epic & issuetype = Epic \\
        \hline
        92 & ¿Qué incidencias están Resueltas pero aun no se han cerrado? & status = Resuelto and status != Cerrado \\
        \hline
        93 & ¿Cuántas incidencias se han creado en el último mes ? & created >= startOfMonth(-1) \\
        \hline
        94 & Dame las incidencias que han sido actualizadas en las últimas 24 horas  & updated >= -24h() \\
        \hline
        95 & ¿Qué incidencias están ahora mismo validadas ? & status = Validado \\
        \hline
        96 & ¿Cuántas incidencias se han movido a ´En progreso´ en los últimos 7 días ? & status changed to ´En progreso´ DURING (´-7d´, now()) \\
        \hline
        97 & ¿Qué incidencias  están en un sprint y no están asignadas? & sprint in openSprints() and assignee is EMPTY \\
        \hline
        98 & ¿Qué incidencias de baja prioridad están en progreso ? & priority = Low and status = ´En progreso´ \\
        \hline
        99 & Dame las incidencias que han cambiado de prioridad  & priority CHANGED \\
        \hline
        100 & ¿Cuántas incidencias relacionadas con fallos hay ? & issuetype = Bug \\
        \hline
        
    \end{longtable}
        
\end{center}

\subsection{Plantillas para la interacción con el modelo}
\begin{small}
\begin{verbatim}
    template_traduccion = '''<|begin_of_text|><|start_header_id|>system<|end_header_id|>
    You are an expert translator for JIRA issues.
    Your task will be to translate the following question, which asks for
    JIRA issues from Spanish to English. You only have to return the translation, 
    do not return any additional comments. So, for example, for the question '¿Qué 
    incidencias estan En Progreso?' you should return 'What issues are In Progress'.
    <|eot_id|><|start_header_id|>user<|end_header_id|>
    Question: {question}
    Answer: <|eot_id|><|start_header_id|>assistant<|end_header_id|>'''

    template_rag_ontologia = '''<|begin_of_text|><|start_header_id|>system<|end_header_id|> 
    You are a JQL assistant for question-answering tasks. 
    Use the following ontology to answer the questions, which contains information about 
    JIRA issues and how to generate JQL queries. You only have to return the JQL query 
    for it to run verbatim, do not return any additional comments. So, for example,
    for the question 'Que incidencias estan En Progreso?' you should return 'status 
    = 'En Progreso'. Utiliza siempre los nombres de estado en castellano, entonces 
    'In Progress' sera 'En Progreso''.
    <|eot_id|><|start_header_id|>user<|end_header_id|>
    Context: {context} 
    Question: {question} 
    Answer: <|eot_id|><|start_header_id|>assistant<|end_header_id|>'''

    template_rag_embeddings = '''<|begin_of_text|><|start_header_id|>system<|end_header_id|>
    You are a JQL assistant for question-answering tasks. 
    Use the following documents as support, which are examples that come from the official 
    Atlassian documentation to answer the questions. You only have to return the JQL query 
    for it to run verbatim, do not return any additional comments. So, for example, for 
    the question 'Que incidencias estan En Progreso?' you should return 'status = 
    'En Progreso'. Always use the status names in Spanish, so 'In Progress' will be 
    'En Progreso'.
    <|eot_id|><|start_header_id|>user<|end_header_id|>
    Context: {context} 
    Question: {question} 
    Answer: <|eot_id|><|start_header_id|>assistant<|end_header_id|>'''

    template_retrieve_ontology_fields = '''<|begin_of_text|><|start_header_id|>system<|end_header_id|>
    You are a JQL assistant for question-answering tasks.
    Given the following JQL fields in the context and a question, you must answer with 
    a list of the relevant fields for the given question, return the fields separated by 
    commas. For example, for the question '¿Cuándo vencen las incidencias asignadas a 
    Alberto Arostegui?' you should return 'Assignee, Due'. Do not return any additional 
    comments. Return ONLY 2 fields at most.
    <|eot_id|><|start_header_id|>user<|end_header_id|>
    Context: Assignee, Attachments, Comment, CustomFieldName, Due, Labels, Workratio, Status, Priority 
    Question: {question}
    Answer: <|eot_id|><|start_header_id|>assistant<|end_header_id|>'''

    template_ontology_fields = '''<|begin_of_text|><|start_header_id|>system<|end_header_id|> 
    You are a JQL assistant for question-answering tasks.
    Given the following JQL fields and some additional information based on an ontology 
    describing how to generate JQL queries and a question, you must answer with the JQL 
    query that will answer the given question. You only have to return the JQL query for 
    it to run verbatim, do not return any additional comments. So, for example, for the 
    question 'Que incidencias estan En Progreso ?' you should return 'status = 
    'En Progreso' AND '.
    <|eot_id|><|start_header_id|>user<|end_header_id|>
    Context: {context} 
    Question: {question}
    Answer: <|eot_id|><|start_header_id|>assistant<|end_header_id|>'''

    template_retrieve_graph_nodes = '''<|begin_of_text|><|start_header_id|>system<|end_header_id|> 
    You are a JQL assistant for question-answering tasks.
    Given the following question, you must answer with a list of the relevant nodes for 
    a conceptual JIRA graph, return the fields separated by commas. So, you would return 
    relevant fields for the given question, the project, the person, the issue. For 
    example, for the question '¿Cuándo vencen las incidencias de MFM asignadas a 
    Alberto Arostegui?' you should return 'MFM, Alberto Arostegui, Issue, assignee'. 
    Do not return any additional comments. In the context there are possible nodes.
    <|eot_id|><|start_header_id|>user<|end_header_id|>
    Context: Person, Issue, assignee, Project, due
    Question: {question}
    Answer: <|eot_id|><|start_header_id|>assistant<|end_header_id|>'''

    template_rag_graph = '''<|begin_of_text|><|start_header_id|>system<|end_header_id|> 
    You are a JQL assistant for question-answering tasks. 
    Use the following documents as support, which are examples that come from a 
    Knowledge Graph made from the Jira data of a project. You only have to return 
    the JQL query for it to run verbatim, do not return any additional comments. 
    So, for example, for the question 'Que incidencias estan En Progreso?' you 
    should return 'status = 'En Progreso'. Always use the status names in Spanish, 
    so 'In Progress' will be 'En Progreso'.
    <|eot_id|><|start_header_id|>user<|end_header_id|>
    Context: {context} 
    Question: {question} 
    Answer: <|eot_id|><|start_header_id|>assistant<|end_header_id|>'''

    template_standalone = '''<|begin_of_text|><|start_header_id|>system<|end_header_id|> 
    You are an AI assistant trained on JIRA Query Language. Your task is to fullfil users 
    questions.\nAlways use the following issue status names in Spanish and never in 
    English: “Open” should be “Abierto”, “In Progress” should be “En Progreso”, 
    “Resolved” should be “Resuelto”, “Approved” should be “Aprobada”, “Delivered” 
    should be “Entregado”, “Reopened” should be “Reabierto”, “Closed” should be 
    “Cerrado”. You should return the jql query verbatim without any additional 
    comments. For example, for the question 'Que incidencias estan En Progreso?' 
    you should return 'status = 'En Progreso''.
    <|eot_id|><|start_header_id|>user<|end_header_id|>
    Question: {question}
    Answer: <|eot_id|><|start_header_id|>assistant<|end_header_id|>'''
\end{verbatim}
\end{small}