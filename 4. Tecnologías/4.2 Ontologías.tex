\subsection{Ontologías}
Una ontología en el ámbito de la informática es una especificación formal de una conceptualización. En otras palabras, es un modelo que representa un conjunto de conceptos dentro de un dominio y las relaciones entre ellos. Estos conceptos son representados por clases, atributos y relaciones. Las ontologías son utilizadas para describir el conocimiento de un dominio y son utilizadas en la inteligencia artificial, la ingeniería del conocimiento y la web semántica.

Como posible base de conocimiento relevante para la generación de consultas JQL se propone crear una ontología que represente las reglas que existen en las consultas JQL. La información que se pretende representar en la ontología se ha extraído directamente de la documentación oficial de Jira, brindada por Atlassian, donde se detallan las reglas que se deben seguir para la creación de consultas JQL \cite{jiradocs}.