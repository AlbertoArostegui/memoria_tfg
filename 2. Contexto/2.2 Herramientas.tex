\subsection{Herramientas}
\subsubsection{JIRA}
JIRA es una herramienta de software propietario desarrollada por Atlassian para coordinar proyectos basados en tareas, llamadas incidencias dentro de la jerga de la aplicación. Esta herramienta sirve tanto para uso interno, como para que acceda el cliente, pudiendo encontrar un punto centralizado donde compartir información sobre el progreso y el estado del proyecto.

Las incidencias son la división atómica de paquetes de trabajo, que representan una tarea cuantificable asignable a un desarrollador y que ayudan a medir el desarrollo llevado a cabo. Al disponer de estados para las incidencias, se puede consultar de manera sencilla cómo progresa el proyecto. 

Dentro de estas se pueden registrar distintos datos, como el tiempo que se prevé que va a tomar la tarea y el tiempo real que toma, mediante registros de trabajo, medidos en horas. Asimismo, se puede incluir información de interés para quien vaya a ser asignado el desarrollo de la incidencia, como una descripción, un resumen o enlaces externos a documentación relevante.

En un proyecto JIRA gestionado en LKS Next-GobTech se gestiona un flujo para las incidencias detallado a continuación: el desarrollador que la realice marcará la incidencia como hecha, a lo que un desarrollador senior validará el trabajo realizado y decidirá si es correcto o si ha de se mejorado. Una vez confirmado, se marcará como validada y podrá pasar a la vista del cliente, que podrá comprobar el trabajo realizado.

\subsubsection{Git - Gitlab}
Al igual que se necesita controlar el estado de trabajos en el proyecto, también es necesario llevar un control de versiones para un óptimo desarrollo de software. En el caso de LKS Next-GobTech se utiliza git \cite{chacon2014progit} como herramienta y Gitlab como punto centralizado donde guardar los repositorios. 

Gitlab es una plataforma que permite gestionar las versiones del software y la colaboración entre desarrolladores. De esta manera, se crea un repositorio para cada proyecto que tiene la empresa y para cada uno de estos repositorios se otorgan permisos de modificación a los desarrolladores que vayan a trabajar en ese proyecto.

Además, se utiliza la integración de JIRA con Gitlab para relacionar las incidencias con cambios realizados en el repositorio asignado al proyecto, de manera que tanto la confirmación del trabajo realizado como del tiempo invertido pueden ser contrastados.