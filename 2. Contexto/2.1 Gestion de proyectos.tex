\subsection{Gestión de proyectos}

La gestión de proyectos es el conjunto de metodologías utilizadas para coordinar la organización, la motivación y el control de recursos con el fin de alcanzar un objetivo. En el caso del desarrollo de software, ha sido un tópico controversial y de relevancia durante su historia, según la conocida ley de Brooks, expuesta en su libro \textit{The mythical man-month} \cite{Brooks1975}, añadir desarrolladores a un proyecto que va detrás del plazo solo hará que se retrase. Dentro de LKS Next-GobTech, donde se coordinan varios proyectos a la vez, resulta crucial una buena organización y división del trabajo en grupos preestablecidos al inicio de estos, con el fin de llevar un óptimo desarrollo en el que se cumplan los plazos establecidos. 

En vistas de lo expuesto, parece obvia la necesidad de una herramienta de software capaz de suplir las necesidades ímplicitas en el desarrollo de software, por lo que dentro de esta empresa, se utiliza una herramienta de software llamada \textit{JIRA}. 