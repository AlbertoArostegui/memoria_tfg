\subsection{JiraGPT Next}
Partiendo del trabajo realizado por Joel García, se dispone de JiraGPT Next como una herramienta que ayuda a recuperar incidencias filtradas utilizando lenguaje natural. De esta manera, una persona que no disponga de conocimiento técnico en la generación de consultas JQL podrá filtrar incidencias facilmente.

Tras esta herramienta se encuentra una llamada de API a un LLM que, utilizando una plantilla para guiar al modelo, pedirá que se traduzca la pregunta en lenguaje natural a una consulta JQL que responda a lo que se pide.

La idea detrás de este nuevo trabajo es realizar un estudio de la mejora de precisión obtenida utilizando arquitecturas RAG.
