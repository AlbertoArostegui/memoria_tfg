\subsection{Implementación de RAG: Knowledge Graphs}
La parte de los grafos de conocimiento supone un reto, ya que es uno de los ejes del trabajo que más dudas ha generado. Para crear un grafo de conocimiento de la información de Jira de LKS Next-GobTech se ha utilizado la API de Jira para extraer la información de los proyectos activos y se ha creado un programa en Python que transforma esa información en un grafo. Se ha utilizado la librería RDFlib para la creación del grafo y se ha almacenado en un fichero en formato RDF. Posterior a esto, para realizar consultas al grafo se ha utilizado la librería SPARQLWrapper, que permite realizar consultas SPARQL al grafo RDF.

Como se 