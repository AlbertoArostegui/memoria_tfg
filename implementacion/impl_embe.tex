\subsection{Implementación de RAG: Embeddings}
Esta propuesta consta de una base de datos vectorial que contiene los embeddings de la documentación oficial de JIRA. Como software se ha optado por ChromaDB~\footnote{https://www.trychroma.com/}, una base de datos vectorial de código abierto que permite realizar búsquedas por similitud de vectores que, además, tiene una sencilla implementación con Langchain.

\subsubsection{Extracción de datos}
Para extraer los datos pertinentes de la documentación oficial de JIRA, se ha construido un \textit{web scraper} que recorre la documentación oficial de JIRA en el lenguaje Python que extrae de las tablas para cada entrada título, descripción y ejemplos. Estos datos se almacenan en un archivo de texto que posteriormente será dividido en diferentes partes para ser procesado por el modelo de embeddings.

Como modelo de embeddings se han utilizado los embeddings de OpenAI, que mediante una llamada con texto devuelven el vector a almacenar en la base de datos. Todo esto con un programa en Python que utiliza Langchain para abstraer el uso de ChromaDB y de la API de OpenAI.

\subsubsection{Interacción con el modelo}
Cuando una pregunta es recibida en el sistema, esta será procesada por los embeddings de OpenAI, lo que genera un vector que se compara con los vectores almacenados en la base de datos vectorial. Los vectores más cercanos a la pregunta serán devueltos, conteniendo la información de Jira pertinente. Esta información se inyecta en el prompt para que el modelo pueda generar una respuesta más precisa.